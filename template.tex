\documentclass[11pt,a4j,uplatex]{jarticle}
\usepackage[dvipdfmx]{graphicx}

\begin{document}

\begin{titlepage}
    \begin{center}
\vspace*{2cm}

\huge{論 文}

\vspace{1cm}

\huge{ホゲホゲに関する研究}

\vspace{5cm}

\LARGE{国際文化学研究科 グローバル文化専攻\\999999c 国際 太郎}

\vspace{4cm}

\LARGE{指導教員:国際 次郎 教授}\\
\LARGE{副指導教員:国際 花子 教授}

    \end{center}
\end{titlepage}

\tableofcontents
\newpage
\begin{abstract}
    ここにはアブストラクトが入ります。ここにはアブストラクトが入ります。ここにはアブストラクトが入ります。ここにはアブストラクトが入ります。ここにはアブストラクトが入ります。ここにはアブストラクトが入ります。ここにはアブストラクトが入ります。ここにはアブストラクトが入ります。ここにはアブストラクトが入ります。ここにはアブストラクトが入ります。ここにはアブストラクトが入ります。ここにはアブストラクトが入ります。ここにはアブストラクトが入ります。ここにはアブストラクトが入ります。ここにはアブストラクトが入ります。ここにはアブストラクトが入ります。ここにはアブストラクトが入ります。ここにはアブストラクトが入ります。ここにはアブストラクトが入ります。ここにはアブストラクトが入ります。ここにはアブストラクトが入ります。ここにはアブストラクトが入ります。ここにはアブストラクトが入ります。ここにはアブストラクトが入ります。ここにはアブストラクトが入ります。ここにはアブストラクトが入ります。ここにはアブストラクトが入ります。ここにはアブストラクトが入ります。
\end{abstract}

\section{はじめに}

    \subsection{研究の背景と目的}

    研究の背景と目的研究の背景と目的研究の背景と目的研究の背景と目的研究の背景と目的研究の背景と目的研究の背景と目的研究の背景と目的研究の背景と目的

    \subsection{論文の構成}

    論文の構成論文の構成論文の構成論文の構成論文の構成論文の構成論文の構成論文の構成論文の構成論文の構成

\newpage

\section{関連研究及び諸概念}

    \subsection{関連研究}
    関連研究関連研究関連研究関連研究関連研究関連研究関連研究関連研究関連研究関連研究関連研究関連研究関連研究


\newpage

\section{使用した技術}

    \subsection{Hoge}
    hogehogehoge

\newpage

\section{システムの概要及び実装}
    \subsection{開発環境}
    開発に使用した環境は以下の通りである。

    PC : MacBook Pro (Retina, 13-inch, Late 2013)

    CPU : 2.6 GHz Intel Core i5

    メモリ : 16 GB 1600 MHz DDR3

    Python : Python 3.6.5

    chainer : Version 5.0.0

    scikit-learn : Version 0.20.0
    \subsection{システムの概要}
    概要概要概要概要概要概要概要概要概要概要概要概要概要概要概要概要

    \subsection{システムの実装}
        \subsubsection{手順1}
        手順1手順1手順1手順1手順1手順1手順1手順1手順1手順1手順1手順1手順1手順1手順1手順1手順1手順1手順1手順1手順1手順1手順1手順1

        \subsubsection{手順2}
        手順2手順2手順2手順2手順2手順2手順2手順2手順2手順2手順2手順2手順2手順2手順2手順2手順2手順2手順2手順2手順2手順2手順2手順2

        \subsubsection{手順3}
        手順3手順3手順3手順3手順3手順3手順3手順3手順3手順3手順3手順3手順3手順3手順3手順3手順3手順3手順3手順3手順3手順3手順3手順3

        \subsubsection{手順4}
        手順4手順4手順4手順4手順4手順4手順4手順4手順4手順4手順4手順4手順4手順4手順4手順4手順4手順4手順4手順4手順4手順4手順4手順4

\newpage
\section{システムの評価}
    \subsection{評価}
    評価評価評価評価評価評価評価評価評価評価評価評価評価評価評価評価評価評価評価評価評価評価評価評価評価評価評価評価

\newpage
\section{結論と今後の課題・展望}

    \subsection{結論}
    結論結論結論結論結論結論結論結論結論結論結論結論結論結論結論結論結論結論結論結論結論結論結論結論結論結論結論結論
    \subsection{今後の課題・展望}
    今後の課題・展望今後の課題・展望今後の課題・展望今後の課題・展望今後の課題・展望今後の課題・展望今後の課題・展望


\newpage
\begin{thebibliography}{15}
    \bibitem{}分類タスクの評価指標の解説とsklearnでの計算方法 (最終閲覧日:2018年12月18日
    https://www.haya-programming.com/entry/2018/03/14/112454
\end{thebibliography}

\newpage
\section*{謝辞}
ほんまにありがとう。


\end{document}
